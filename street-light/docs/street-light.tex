\documentclass{article}

\usepackage{titling}
\usepackage{hyperref}
\usepackage{listings}
\usepackage{xcolor}

\renewcommand{\contentsname}{Contenidos}

\hypersetup{
    colorlinks=true,
    linkcolor=blue,
    filecolor=magenta,
    urlcolor=blue,
}

\urlstyle{same}

\lstdefinelanguage{atmegaAsm}{
    sensitive=false,
    alsoletter={\.},
    keywordsprefix={\#},
    comment=[l]{;},
    % Instructions
    keywords=[1]{
        ldi, lds, ld, sts, out, sbi, sei, tst, brne, dec, inc, sbrc, in, andi,
        rjmp, push, pop, add, reti
    },
    % Directives
    keywords=[2]{
        .text, .global,
    },
    % Registers
    keywords=[3]{
        r16, r17, r20, r21, r22, DDRB, PORTB, TCCR0B, TCNT0, TIMSK0, SREG,
    }
}

\definecolor{mGreen}{rgb}{0,0.6,0}
\definecolor{mGray}{rgb}{0.5,0.5,0.5}
\definecolor{mPurple}{rgb}{0.58,0,0.82}
\definecolor{backgroundColour}{rgb}{0.95,0.95,0.92}

\lstdefinestyle{atmegaAsm}{
    backgroundcolor=\color{backgroundColour},
    commentstyle=\color{mGreen},
    keywordstyle=\color{magenta},
    numberstyle=\tiny\color{mGray},
    stringstyle=\color{mPurple},
    basicstyle=\footnotesize,
    breakatwhitespace=false,
    breaklines=true,
    captionpos=b,
    keepspaces=true,
    numbers=left,
    numbersep=5pt,
    showspaces=false,
    showstringspaces=false,
    showtabs=false,
}

\begin{document}

\title{Actividad 1 - Semáforo}
\author{Moltrasio, Mauro Ezequiel}
\date{}
\renewcommand{\abstractname}{\vspace{-\baselineskip}}

\begin{titlingpage}
    \maketitle
    \begin{abstract}

        Clase: Estructura de computadores.

        Profesor: José Delgado Pérez.
    \end{abstract}
\end{titlingpage}

\tableofcontents

\section{Introducción}

En este documento se describirán los pasos timados para la codificación de un
semáforo con una placa Arduino uno. El lenguaje de prigramación elegido será
assembly y se utilizará PlatformIO para simplificar la compilación y carga del
código a la placa.

\section{Primeros pasos}

Tras instalar PlatformIO, se procedió a utilizar un código de ejemplo con un
LED titilando (blinky) para corroborar que todo funcionara correctamente.
El código de ejemplo utilizado como punto de inicio se puede encontrar en el
siguiente link: \url{https://gitlab.com/jjchico-edc/avr-bare/-/blob/master/projects/01-blink-asm/main.S}

Si bien se podría utilizar la función wait provista en ese ejemplo para
codificar el semáforo, una forma más sencilla es utilizar un timer, por lo que
se decidió cambiar por este enfoque.

\section{Configurando Timer 0}

Se eligió utilizar el timer 0 del procesador, ya que es un timer de uso
general y nos permitirá crear una base de tiempo sobre la cual basar nuestra
máquina de estados (más sobre la máquina de estados en las siguientes
secciones).

Para experimentar con el timer 0, primero se procedió a reemplazar la función
wait del ejemplo blinky con el flag de overflow de dicho timer. Este flag es
el bit TOV0 del registro TIFR0.

Con el objetivo de tener un blink que apenas sea perceptible al ojo humano, se
configuraron los bits CS[0:2] del registro TCCR0B para utilizar una base de
tiempo de fclk/1024 (valor 0x05). Este valor da una frecuencia de
16MHz / 1024 = 15,625kHz. Esto combinado con la cuenta de 8 bits del timer nos
da un blink de aproximadamente 61Hz, lo cual es apenas visible, pero suficiente
para poder ver el timer funcionando.

Finalmente, se cambió de utilizar el chequeo del flag TOV0 en línea con el lazo
principal a utilizar la interrupción de overflow del timer. Esto se consigue
seteando el bit TOIE0 del registro TIMSK0, usando la instrucción SEI para
habilitar las interrupciones globales y definiendo la rutina de manejo de
interrupcion TIMER0{\_}OVF{\_}vect. Cabe aclarar que esta última rutina viene
provista por el framework avr-libc que utiliza PlatformIO, caso contrario
deberíamos proveer la instrucción jmp correcta en el vector de interrupciones.

Con la interrupción del timer funcionando, estamos en condiciones de plantear
la resolución del semáforo.

\section{Resolución de un semáforo}

La lógica de un semáforo se puede considerar como una máquina de estados con
4 estados (enumerados en binario):

\begin{itemize}
    \item 00: Luz roja encendida.
    \item 01: Luz amarilla encendida, cambiando a verde.
    \item 10: Luz verde encendida.
    \item 11: Luz amarilla encendida, cambiando a rojo.
\end{itemize}

Podemos ver un par de patrones en la lista anterior.
En principio, todos los estados se pueden visitar simplemente incrementando el
valor de la variable que contiene el estado.
También se ve que el bit menos significativo nos define si el semáforo tiene
encendida la luz amarilla o si está en verde o rojo y en caso de estar a 0,
el segundo bit nos define que estamos en rojo cuando está a 0 y verde cuando
está a 1.
Todo esto nos será de utilidad a la hora de plantear el código.

Sabiendo cómo será la máquina de estado, debemos primero definir una base de
tiempo adecuada para poder iterar sobre los estados. Utilizando el timer 0
que se describió en la sección anterior que tiene una frecuencia de cuenta de
15,625 kHz, se decidió crear una base de tiempo de 10ms, la cual es pequeña,
pero nos será de utilidad en futuras actividades para crear funciones de
antirrebote. Para que la interrupción del timer se dispare cada 10ms,
si una cuenta es de 1/15,625kHz = 64us = 0.064ms, entonces se necesitan
156,25 cuentas. Podemos redondear a 156 y como el timer cuenta hasta 255,
el registro TCNT0 deberá recargarse con el valor 255 - 156 = 99 cada vez que
se genere la interrupción. Esta base de tiempo se puede complementar con otro
contador por software para generar bases más largas.

Con la base de tiempo definida y una primera aproximación de la máquina de
estados, podemos empezar a definir dónde vivirán algunas de las variables
necesarias. Dada la simplicidad de la aplicación y la gran cantidad de
registros disponibles en el procesador, se dicidió utilizar algunos de estos
para definir variables directamente, especificamente:

\begin{itemize}
    \item R16 y R17: Se utilizarán como registros generales para operaciones varias.
    \item R20: Mantendrá una base de tiempo para los cambios de estado del semáforo.
    \item R21: Trabajará en conjunto con R20 para tener distintos tiempos en los distintos estados.
    \item R22: Almacenará el estado de nuestro semáforo propiamente dicho.
\end{itemize}

También podemos definir los pines que se utilizarán para manejar las distintas
luces:

\begin{itemize}
    \item PB0: Luz roja para autos.
    \item PB1: Luz amarilla para autos.
    \item PB2: Luz verde para autos.
    \item PB3: Luz roja para peatones.
    \item PB4: Luz verde para peatones.
\end{itemize}

Aún no hemos hablado sobre cómo será el manejo de las luces peatonales, pero si
planteamos una máquina de estados similar a la del semáforo de autos:

\begin{itemize}
    \item 00: Luz peatonal verde encendida.
    \item 01: Luz peatonal verde encendida.
    \item 10: Luz peatonal roja encendida.
    \item 11: Luz peatonal roja encendida.
\end{itemize}

Vemos que en realidad sólo tiene 2 estados, pero estos se pueden duplicar para
que coincidan con nuestra máquina de estado original. Por lo tanto, en lugar
de plantear otra máquina de estado, el código utilizará una única máquina y
controlará ambos semáforos de forma adecuada en base al estado en el que se
está.

En este punto tenemos todos los elementos necesarios para plantear la
resolución al problema.

\section{Código final}
\lstinputlisting[style=atmegaAsm,language=atmegaAsm]{../src/main.S}

\end{document}
