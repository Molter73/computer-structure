\documentclass{article}

\usepackage{titling}
\usepackage{hyperref}
\usepackage{listings}
\usepackage{xcolor}

\renewcommand{\contentsname}{Contenidos}

\hypersetup{
    colorlinks=true,
    linkcolor=blue,
    filecolor=magenta,
    urlcolor=blue,
}

\urlstyle{same}

\lstdefinelanguage{atmegaAsm}{
    sensitive=false,
    alsoletter={\.},
    keywordsprefix={\#},
    comment=[l]{;},
    % Instructions
    keywords=[1]{
        ldi, lds, ld, sts, out, sbi, sei, tst, brne, dec, inc, sbrc, in, andi,
        rjmp, push, pop, add, reti, cbi, sbrs, lsl, sbis, sbr,
    },
    % Directives
    keywords=[2]{
        .text, .global,
    },
    % Registers
    keywords=[3]{
        r16, r17, r20, r21, r22, r23, DDRB, PORTB, TCCR0B, TCNT0, TIMSK0, SREG,
        DDRD, PORTD, PB0, PB1, PB2, PB3, PB4, PB5, PD2, SPH, SPL,
    }
}

\definecolor{mGreen}{rgb}{0,0.6,0}
\definecolor{mGray}{rgb}{0.5,0.5,0.5}
\definecolor{mPurple}{rgb}{0.58,0,0.82}
\definecolor{backgroundColour}{rgb}{0.95,0.95,0.92}

\lstdefinestyle{atmegaAsm}{
    backgroundcolor=\color{backgroundColour},
    commentstyle=\color{mGreen},
    keywordstyle=\color{magenta},
    numberstyle=\tiny\color{mGray},
    stringstyle=\color{mPurple},
    basicstyle=\footnotesize,
    breakatwhitespace=false,
    breaklines=true,
    captionpos=b,
    keepspaces=true,
    numbers=left,
    numbersep=5pt,
    showspaces=false,
    showstringspaces=false,
    showtabs=false,
}

\begin{document}

\title{Actividad 2 - Semáforo con pulsador}
\author{Moltrasio, Mauro Ezequiel}
\date{}
\renewcommand{\abstractname}{\vspace{-\baselineskip}}

\begin{titlingpage}
    \maketitle
    \begin{abstract}

        Clase: Estructura de computadores.

        Profesor: José Delgado Pérez.
    \end{abstract}
\end{titlingpage}

\tableofcontents

\section{Introducción}

Este documento describe una continuación a la actividad 1, se aprovechó la
mayor parte del código de dicha actividad y no se volverá a describir el
proceso para describir cómo se llegó a ese punto. En lugar de esto, se
describirán los cambios realizados a ese código para resolver la actividad de
semáforo con pulsador para peatones.

\section{Manejo de un pulsador}

La primer gran diferencia con la actividad anterior viene clara desde el
enunciado, debemos manejar un pulsador. Con este propósito, se eligió el pin 2
del puerto D como nuestra entrada y se la configuró para que utilice el
resistor de pull-up interno del microcontrolador, simplificando el circuito
del pulsador, sólo es necesario conectar un extremo del pulsador a tierra y el
otro al pin correspondiente a PD2. Se debe tener en cuenta que en esta
configuración, PD2 se pondrá a 0 cuando el pulsador esté presionado.

Debido que en la vida real las condiciones no son perfectas y el ruido ambiente
puede afectar las señales de nuestro microcontrolador, se implementará una
función de antirrebote para evitar falsas activaciones.

Aprovechando el manejador de interrupciones del timer 0 definido en la
actividad anterior (interrupción por overflow cada 10ms), resulta trivial
crear una rutina de antirrebote. Simplemente seleccionamos un registro, en
nuestro caso r23, y en cada interrupción rotamos su contenido un bit a la
izquierda, luego leemos el pin PD2 y si el pulsador se encuentra presionado
colocamos un 1 en el bit menos significativo del registro, caso contrario
dejamos un 0. Una vez que el registro se encuentre completamente lleno de 1s,
significa que el pulsador se encontró presionado durante al menos 80ms y
podemos asumir que no es una falsa activación.

\section{Adaptación del código de semáforo}

En la actividad anterior teniamos una máquina de estado para el semáforo
relativamente sencilla:

\begin{itemize}
    \item 00: Luz roja encendida.
    \item 01: Luz amarilla encendida, cambiando a verde.
    \item 10: Luz verde encendida.
    \item 11: Luz amarilla encendida, cambiando a rojo.
\end{itemize}

Esta máquina de estado aún nos es útil, una vez el pulsador se presione se
deberá proceder a realizar un recorrido completo por esta máquina de estado.
De esta forma, podríamos considerar que se agrega un estado más a nuestra
máquina.

\begin{itemize}
    \item 0xx: Luz verde encendida, esperando pulsador.
    \item 100: Luz verde encendida.
    \item 101: Luz amarilla encendida, cambiando a rojo.
    \item 110: Luz roja encendida.
    \item 111: Luz amarilla encendida, cambiando a verde.
\end{itemize}

Los estados se reacomodaron ligeramente para simplificar la lógica del
programa, pero se puede ver que nuestra máquina de estado original vive como
un subestado de una nueva máquina en la que tenemos 2 estados:

\begin{itemize}
    \item Esperar presión del pulsador.
    \item Realizar un ciclo de semáforo y volver a estado anterior.
\end{itemize}

Con esto se puede ver que los cambios al código original serán mínimos, sólo
debemos tener un lazo pequeño en nuestro programa principal que espera el
pulsador, una vez presionado se utilizará otro lazo que realizará el ciclado
de semáforo y al terminar volverá al lazo que espera el pulsador.

\section{Código final}
\lstinputlisting[style=atmegaAsm,language=atmegaAsm]{../src/main.S}

\end{document}
